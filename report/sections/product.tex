\section{Code Documentation}
This section will go through the different futhark code files and document their general content, and go into detailed read-write analysis of the functions.
\subsection{framework/guess.fut}
This file contains several mostly photosynthesis-related enumerators and constants, followed by class definitions for PhotosynthesisStresses, PhotosynthesisResult, PhotosynthesisEnvironment, Pft and their member methods.
\subsection{framework/guessmath.fut}
Guessmath contains a number of simple functions and mathematical constants.
\subsection{framework/parameters.fut}
Parameters contains various enumerators and a number of parameters for the overall model.
\subsection{modules/canexch.fut}
This file contains a number of constants, and then a number of functions. Most of the heavy lifting is done while 'in' the functions of this file, as per the initial gprof measurements.

\subsubsection{alphaa()}
\begin{minted}{cpp}
double alphaa(const Pft& pft)
\end{minted}

Receives a Pft by reference, checks the phenology and ifnlim fields, returns one of a number of constants. Does not modify the Pft.
\subsubsection{vmax()}

\begin{minted}{cpp}
void vmax(double b,
          double c1,
          double c2,
          double apar,
          double tscal,
          double daylength,
          double temp,
          double nactive,
          bool ifnlimvmax,
          double& vm,
          double& vmaxnlim,
          double& nactive_opt)
\end{minted}

All the reference-passed parameters are written before ever being read, and are thus candidates for privatization.

\subsubsection{photosynthesis()}
\begin{minted}{cpp}
void photosynthesis(const PhotosynthesisEnvironment& ps_env,
                    const PhotosynthesisStresses& ps_stresses,
                    const Pft& pft,
                    double lambda,
                    double nactive,
                    double vm,
                    PhotosynthesisResult& ps_result)
\end{minted}
The fields of ps_result are always before having been read, and it is thus privatizable. The rest are all consts, and therefore are not modified.

\begin{minted}{futhark}
let photosynthesis(ps_env : PhotosynthesisEnvironment,
                  ps_stresses : PhotosynthesisStresses,
                  pft: Pft,
                  lambda : real,
                  nactive : real,
                  vm : real)
                  : PhotosynthesisResult =
\end{minted}

It calls the functions: negligible(), alphaa, readQ10() (see the relevant section), exp(), min(), vmax(), PhotosynthesisResult()

In the input objects, the function reads the following fields:

ps_env.temp
ps_env.co2
ps_env.fpar
ps_env.par
ps_env.daylength

ps_stresses.ifnlimvmax
ps_stresses.moss_ps_limit
ps_stresses.graminoid_ps_limit
ps_stresses.inund_stress

pft.pstemp_max
pft.pstemp_min
pft.pathway
pft.lifeform


\subsubsection{assimilation_wstress()}
\begin{minted}{cpp}
void assimilation_wstress(const Pft& pft,
      double co2,
      double temp,
      double par,
      double daylength,
      double fpar,
      double fpc,
      double gcbase,
      double vmax,
      PhotosynthesisResult& phot_result,
      double& lambda,
      double nactive,
      bool ifnlimvmax,
      double moss_wtp_limit,
      double graminoid_wtp_limit,
      double inund_stress)
\end{minted}
phot_result is always written before being read, and is thus privatizable. The same goes for lambda. Thus we get this signature:

\begin{minted}{futhark}
let assimilation_wstress
    (pft: Pft,
    co2: real,
    temp: real,
    par: real,
    daylength: real,
    fpar: real,
    fpc: real,
    gcbase: real,
    vmax: real,
    nactive: real,
    ifnlimvmax: bool,
    moss_wtp_limit: real,
    graminoid_wtp_limit: real,
    inund_stress: real)
    : (Pft, PhotosynthesisResult, real) =
\end{minted}

The function calls negligible(), PhotosynthesisResult(), photosynthesis() and abs(). For testing purposes, it reads the following fields from the input object: pft.lambda_max,

and it also reads, via the photosynthesis() function,

pft.pstemp_max
pft.pstemp_min
pft.pathway
pft.lifeform

\subsection{modules/q10.fut}
This file contains the definition and operator for a lookup table for some pre-calcuated values, and some relevant constants.

\subsection{modules/soil.fut}
This file contains a large number of constants.
