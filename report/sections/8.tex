\section{Current state and a few recommendations}
Currently there is a use-after-consume\cite{bug1} bug in the implementation of the \texttt{canopy_exchange()} function. Normally these can be fixed using \texttt{copy}, but this does not seem to be the case.

The code relies on some global constants that are set in the parameter files, and in the \texttt{global.ins} file.

In several places in the code, \texttt{TODO}, \texttt{FIXME} etc are used to denote this and other issues. A couple of functions have yet to be implemented for \texttt{canopy_exchange} to work, specifically
\texttt{irrigated_water_uptake}, \texttt{water_uptake_twolayer} and \texttt{water_uptake}. The first of these has a few dependencies of its own in the soil module.

I would also recommend writing functions for the testing framework that just print literals of entire objects. This will make it much more straightforward to later on begin testing arrays of objects, and it will be less error prone, as one will no longer have to keep track of data dependencies of subfunctions of subfunctions, although writing these print functions will require some investment up front.

After finishing \texttt{canopy_exchange} with its dependencies, one can move on to the other functions in the main outer loops.
