\section{Translation work}
In order for the resulting conventions and code of the present project to be made the most useful for a follow up project, certain considerations are relevant to take into account.

There is a significant difference between writing Futhark code from the bottom up, and to writing it as a translation from an object oriented language - C++ in this case. This author does have some experience in writing Futhark programs, and likewise in writing C and C\# programs, but not much experience in translating code in general, even less so in translating idiomatic code between paradigms.

It seems plausible that a hypothetical continuator of the present work is in a similar situation, and for that reason, I will present the reasoning behind the present work for consideration.
\subsection{Data organisation}
In translating the architecture, several choices presented themselves. First off, preserving the folder and file structure in translating the framework seems to have no drawbacks, and is very useful for keeping track of progress, comparing the translated code with its referent, etc. Given the dependency segregation principle of the SOLID principles of OOP, C++ has a distinction between header files (.h) and implementation files (.cpp) which is absent in futhark. Header files contain the definitions or interfaces of functions and classes, and the implementation files contain their implementations. This is not a strict distinction - the compilers will not force you to actually structure the code in this way - but it will pair headers and their corresponding implementation files together when compiling and linking the code.

Because of this one-to-one correspondece headers have with their bodies, I decided to conceptually combine them into single futhark files. So, for canexch.h and canexch.cpp, only canexch.fut will exist. For the purpose of keeping track of progress and such, the content from the header file should appear first, and their sepparation should be denoted by a visually substantial comment.

Whenever there is a name for something, a file, field or function, or anything else, it is desirable to preserve this name in the translation. When functions are overloaded, or have default values, and in many other situations, a reasonable alternative must be found.

\subsection{Branches}
\begin{minted}{cpp}
if (!negligible(plai_grass))
  flai=indiv.lai_today()/plai_grass;
else
  flai=1.0;
\end{minted}

Becomes

\begin{minted}{futhark}
let flai =
  if (!negligible(plai_grass)) then
    lai_today(indiv, pfts[indiv.pft_id], ppfts[indiv.pft_id])/plai_grass
  else 1.0
\end{minted}
Instead of a conditional branching to two assignments, we have one assignment to the outcome of a conditional, which just branches to two values. This is both easy to read and to write.

\subsection{Basic values and operations}
Most of the basic arithmetical operations used in the framework are common to both languages, as are basic data types. Doubles from C++ are represented by reals, which are parameterised and may be f64s or f32s for instance. Similar goes for ints. C++ structs can be represented most simply using records, and have almost equivalent semantics.

In many places of the code, small arrays of constant length are present. This is considered harmful\cite{smallarraysbad}, and is a possible target for future optimization.

\subsection{Enumerators}
In C++, enumerators are types with a very limited number of possible values, which are named tokens whose underlying representation is of secondary importance. However, they are essentially represented as integer constants, with each value of the enumerator, in order, having a value starting at zero and going up. For this reason, the last value of the enumerator is sometimes a pseudo-value, called for instance, NUM_SOME_ENUMERATOR. This allows the programmer to reference the size of the enumerator space, so to speak, but to me it seems like a hack. Printing an enumerator value to standard out will yield its underlying integer representation.

Futhark has built-in enumerator types which can contain arbitrary data. This is very flexible, however they can not be type-unified with integers. I went back and forth between the elegant native futhark representation, and the unelegant integer constant representation, but due to the seeming infeasability of printing the code representation of an enumerator in C++, i have currently gone for the integer constant representation. While this unfortunately is slightly more verbose in the futhark source code, it makes testing much more streamlined and productive, as one does not have to write unique print functions for each enumerator.

\subsection{Functions, calling conventions and memory management}
A widespread and useful convention in C++ is to have void functions that do not return anything, and that instead have return values of functions be allocated by the caller and passed to the callee by reference. C++ objects are likewise always passed by reference implicitly, whereas structs are passed by value unless otherwise specified.

Futhark passes everyting as values, and has no concept of reference or manual memory management, such as the C++ destinction between declaration and initialization.

Write-first values should be privatized, and only those values which have been modified should be returned. These values should then be shadowed in the caller, syntactically simulating them having been overwritten (which is morally impossible in a pure functional language).

\subsection{Classes and their objects}
Classes and objects do not exist in Futhark, so there is no one-to-one translation to be had. A reasonable convention therefore has to be discovered. In general, it is not desired for the translation to be a simulation of the original, but a functionally equivalent implementation of the original in the target language, true to the target language. One must therefore distinguish between objects and classes that exist to satisfy the OO paradigm, and objects and classes that exist to represent something in the mathematical model, of which the framework is an implementation.

To keep things simple, classes are represented as records, member methods as functions on records, and constructors as functions returning records.

Member functions should refer to the 'object' they are operating on as 'this'.

Class definitions:
\begin{minted}{c++}
class SomeClass {
public:
  bool somefield;
  double someotherfield;
  // all the other class members here
}
\end{minted}

\begin{minted}{futhark}
type SomeClass = {
  somefield: bool,
  someotherfield: real
}
\end{minted}

Constructors:
\begin{minted}{c++}
public
  SomeClass() {
    somefield = true;
    someotherfield = 2.0;
  }
\end{minted}

\begin{minted}{futhark}
let SomeClass() : SomeClass =
  {somefield = true
  ,someotherfield = 2.0
  }
\end{minted}

Method:
\begin{minted}{c++}
double someMethod() {
  if(somefield){
    return someotherfield;
  } else {
    return someotherfield/1.0;
  }
}
\end{minted}
\begin{minted}{futhark}
let someMethod(this: SomeClass) : SomeClass =
  if(this.somefield)
    then this.someotherfield
    else this.someotherfield/1.0
\end{minted}

Another method:
\begin{minted}{c++}
void anotherMethod(double somevalue) {
  someotherfield *= somevalue;
}
\end{minted}
\begin{minted}{futhark}
let anotherMethod(this: SomeClass, somevalue: real) : SomeClass =
  this with someotherfield = someotherfield * somevalue
\end{minted}

Notice the verbosity of the in-place update of records. This is mostly an aesthetic concern, but when we get to functions that modify many fields of records many times, this becomes an issue, which brings us to:

\subsection{Handling many in-place updates}
OOP generally involves heavy modification of fields of objects, and a translation to futhark easily ends up having an equal number of in-place updates to records, which ends up being difficult to read.

It is advisable to extract values that are to be read multiple times into local variables, and it is advisable to generate and modify fields as local variables, externally, and then doing a series of in-place updates or a literal record construction at the end of the function. This keeps the code cleaner and more easily readable. Example, compare the following snippets for readability:
\newpage
\begin{minted}{futhark}
-- 1) Inundation stress
let ps_result = ps_result with agd_g = (ps_result.agd_g * inund_stress)
let ps_result = ps_result with rd_g = (ps_result.rd_g * inund_stress)

-- 2a) Moss dessication
let ps_result = if (pft.lifeform == MOSS)
  then
    let ps_result = ps_result with agd_g = (ps_result.agd_g * moss_ps_limit)
    let ps_result = ps_result with rd_g = (ps_result.rd_g * moss_ps_limit)
    in ps_result
  else ps_result
\end{minted}

Compare the readability of the above code to:

\begin{minted}{futhark}
-- 1) Inundation stress
let agd_g = agd_g * inund_stress
let rd_g = rd_g * inund_stress

-- 2a) Moss dessication
let (agd_g, rd_g) = if (lifeform == MOSS)
  then (agd_g * moss_ps_limit, rd_g * moss_ps_limit)
  else (agd_g, rd_g)
\end{minted}

The first example (point 1) should suffice to motivate just avoiding doing inplace updates as much as possible, but notice how it becomes especially unwieldy when doing multiple conditional updates.

\begin{minted}{futhark}
-- returning after having done verbose inplace updates
  in ps_result

-- returning by constructing a literal of the record
  in {agd_g = agd_g, rd_g = rd_g}

-- returning by constructing a literal of the record
  let ps_result = ps_result with agd_g = agd_g
  in ps_result with rd_g = rd_g
\end{minted}

No matter the size of the record, constructing a literal is the simplest way to do it. The only case where it is advisable to do inplace updates before returning is if there are a lot of fields with values that have not been modified, and which should be preserved. Only in that case would you end up with shorter and more readable code by doing inplace updates.

However, that is actually what is happening most of the time - most of the time, you are modifying a few fields in very large records. Futhermore, keeping track of which fields have been extracted and must be put back is errorprone. Doing inplace updates will not harm the performance of the code, since records are just sugar that disappears.

\subsection{Updating arrays in records}
Consider the following record:

\begin{minted}{futhark}
type Fluxes = {
  annual_fluxes_per_pft : [npft][NPERPFTFLUXTYPES]real,
  monthly_fluxes_patch : [12][NPERPATCHFLUXTYPES]real,
  monthly_fluxes_pft : [12][NPERPFTFLUXTYPES]real,
  daily_fluxes_patch : [365][NPERPATCHFLUXTYPES]real,
  daily_fluxes_pft : [365][NPERPFTFLUXTYPES]real
}
\end{minted}

Lets say we want to update three of these inner arrays. Ideally, we would like a simple syntax like this:

\begin{minted}{futhark}
let report_flux_PerPFTFluxType_verbose2
    ( fluxes: Fluxes
    , flux_type: PerPFTFluxType
    , pft_id: int
    , value: real
    , month: int
    , day: int) : Fluxes =
      let fluxes.annual_fluxes_per_pft[pft_id, flux_type] += value
      let fluxes.monthly_fluxes_pft[month, flux_type] += value
      let fluxes.daily_fluxes_pft[day, flux_type] += value
      in fluxes
\end{minted}

Hower, it is not possible to do this. Ditching the addition-assignment operator, and using the current \texttt{with} syntax, we get the following error:

\begin{minted}{bash}
Types
  [dim₁][dim₂]src₀
and
  {annual_fluxes_per_pft: [npft][NPERPFTFLUXTYPES]f64,
   daily_fluxes_patch: [365][NPERPATCHFLUXTYPES]f64,
   daily_fluxes_pft: [365][NPERPFTFLUXTYPES]f64,
   monthly_fluxes_patch: [12][NPERPATCHFLUXTYPES]f64,
   monthly_fluxes_pft: [12][NPERPFTFLUXTYPES]f64}
do not match.
\end{minted}

adding some parenthesis solve this error, and we get the following rendering to the same effect as the one intended above. Lets also just look at the function body, since the signature is the same:

\begin{minted}{futhark}
let fluxes = fluxes with annual_fluxes_per_pft =
  (fluxes.annual_fluxes_per_pft with [pft_id, flux_type] =
    fluxes.annual_fluxes_per_pft[pft_id, flux_type] + value)
let fluxes = fluxes with monthly_fluxes_pft =
  (fluxes.monthly_fluxes_pft with [pft_id, flux_type] =
    fluxes.annual_fluxes_per_pft[pft_id, flux_type] + value)
let fluxes = fluxes with daily_fluxes_pft =
  (fluxes.daily_fluxes_pft with [day,flux_type] =
    fluxes.daily_fluxes_pft[day,flux_type] + value)
in fluxes
\end{minted}

This is a lot less readable, however, it is still not permitted. We get the following error:
\begin{minted}{futhark}
Error at example.fut:218:8-13:
Using variable "fluxes", but this was consumed at 219:8-220:54.  (Possibly through aliasing.)
\end{minted}
This is a very interesting error. The inner update shadows the definition of fluxes, which can then not be shadowed in the outer update, it seems. This might also explain the size-type error. The language does not seem to expect records of a depth greather than one, or to support these. Two solutions considered are to forget in-place updates and just construct a new record, in this manner:

\begin{minted}{futhark}
{ annual_fluxes_per_pft =
    annual_fluxes_per_pft with [pft_id, flux_type] =
      annual_fluxes_per_pft[pft_id, flux_type] + value
, monthly_fluxes_patch
, monthly_fluxes_pft =
    monthly_fluxes_pft with [month, flux_type] =
      monthly_fluxes_pft[month, flux_type] + value
, daily_fluxes_patch
, daily_fluxes_pft =
    daily_fluxes_pft with [day,flux_type] =
      daily_fluxes_pft[day,flux_type] + value
}
\end{minted}
This will work, but it results in very verbose code for larger records.

The solution chosen is to use \texttt{copy} whenever there is a consumption-related error, in the following manner:

\begin{minted}{futhark}
let fluxes = fluxes with annual_fluxes_per_pft =
  (copy fluxes.annual_fluxes_per_pft with [pft_id, flux_type] =
    fluxes.annual_fluxes_per_pft[pft_id, flux_type] + value)
let fluxes = fluxes with monthly_fluxes_pft =
  (copy fluxes.monthly_fluxes_pft with [pft_id, flux_type] =
    fluxes.annual_fluxes_per_pft[pft_id, flux_type] + value)
in fluxes with daily_fluxes_pft =
  (copy fluxes.daily_fluxes_pft with [day,flux_type] =
    fluxes.daily_fluxes_pft[day,flux_type] + value)
\end{minted}

This gets the code past the checker. The danger then is that this might lead to redundant copying. If we compile this version and compare it to the one where we construct a literal of an updated record, using \texttt{futhark dev -s test.fut}, we can see that the two versions compile to effectively identical code. This is probably because intitial optimization steps of the compiler will recognize the copying step as redundant, and optimize it away. Since we are shadowing a value using a modified copy of itself, the original value will never be referenced again, only the modified copy. The optimizer sees this, and just modifies the original data instead of the copy.

\subsection{A common map reduce mattern}

Consider the following common pattern. We modify each member of a collection of values, and we perform some kind of accumulation over the modified values:
\begin{minted}{cpp}
double fpar_tree_total=0.0;

vegetation.firstobj();
while (vegetation.isobj) {
  Individual& indiv=vegetation.getobj();

  // For this individual ...

  if (indiv.pft.lifeform==GRASS || indiv.pft.lifeform==MOSS) {

    // ...

    indiv.fpar=max(0.0,fpar_grass*flai-max(fpar_ff*flai,fpar_min));

    // ..
  }

  if (indiv.pft.lifeform==TREE) fpar_tree_total+=indiv.fpar;

  vegetation.nextobj();
}
\end{minted}

We can translate this to a map followed by a reduce. However, we have to bind the result of the map, to shadow the old values, so the changes persist. Simply piping the result into the reduction will not make the modifications persistent:

\begin{minted}{futhark}
let (vegetation) = map (\indiv ->
  if (pfts[indiv.pft_id].lifeform==GRASS || pfts[indiv.pft_id].lifeform==MOSS) then

    -- ...

    let indiv = indiv with fpar = max(0.0,fpar_grass*flai-max(fpar_ff*flai,fpar_min))

    -- ...

    in indiv
  else (indiv)
  ) vegetation
let fpar_tree_total : real =
  reduce (+) 0.0 <| map
    (\i -> if pfts[i.pft_id].lifeform==TREE then i.fpar else 0.0
    ) vegetation
\end{minted}

\subsection{Type checker helpers}
Consider the \texttt{init_canexch} function:
\begin{minted}{futhark}
let init_canexch(patch : Patch, climate : Climate, vegetation : [npft]Individual, date : Date, pfts: [npft]Pft) : (Patch, Vegetation) =
  let vegetation =
    if (date.day == 0) then
      map (\indiv ->
        let indiv = indiv with anpp = 0.0
        let indiv = indiv with leafndemand = 0.0
        let indiv = indiv with rootndemand = 0.0
        let indiv = indiv with sapndemand = 0.0
        let indiv = indiv with storendemand = 0.0
        let indiv = indiv with hondemand = 0.0
        let indiv = indiv with nday_leafon = 0
        let indiv = indiv with avmaxnlim = 1.0
        let indiv = indiv with cton_leaf_aavr = 0.0
        in
        if (!negligible(indiv.cmass_leaf) && !negligible(indiv.nmass_leaf)) then
          indiv with cton_leaf_aopt = indiv.cmass_leaf / indiv.nmass_leaf
        else
          let indiv = indiv with cton_leaf_aopt = pfts[indiv.pft_id].cton_leaf_max
          let indiv = indiv with mlai = replicate 12 realzero
          let indiv = indiv with mlai_max = replicate 12 realzero
          in indiv
      ) vegetation
    else vegetation
  in (patch with wdemand_day = 0, vegetation)
\end{minted}
This implementation fails with the following error: (referencing the first inner binding)
\begin{minted}{bash}
Error at futsource/modules/canexch.fut:1674:21-41:
Full type of
  indiv
is not known at this point. Add a type annotation to the original record to
disambiguate.
\end{minted}

If we modify the line with the \texttt{map} in this manner, to add a type annotation, in this manner:

\begin{minted}{futhark}
map (\indiv : Individual ->
\end{minted}

We still get the same error. This same error pattern is present in many places in the code. The current solution is to write functions of this kind:

\begin{minted}{futhark}
let type_checker_helper8 (i: Individual) : Individual = i
\end{minted}

And applying it in this manner:

\begin{minted}{futhark}
let vegetation =
  if (date.day == 0) then
    map (\indiv ->
      let indiv = type_checker_helper8(indiv)
      let indiv = indiv with anpp = 0.0
      let indiv = indiv with leafndemand = 0.0
\end{minted}

It seems that it fails to derive the type of the mapped individual from the collection mapped over, at least in these kinds of cases. The outer array being mapped over is type annotated in the function signature, so there should be no real ambiguity.
