\section{Compiler}
In this section, we will go into details on the implementation of our Futhark code generation.  The code generation consists of several preprocessing steps, two of which are optional, and then two different code generators.  The first preprocessing stage is caramelization (desugaring).  Since many of the constructs in our language can be expressed in terms of others, we deemed it prudent to do this, so as to reduce the complexity and semantic overlap of different parts of the code generators.  The second is optimization, which combines some \tt{LFun}s and reduces them when possible to simpler and more regular structures without changing the semantics.  The third preprocess stage is the
'matrix' stage, which expresses the \tt{LFun}s as matrices, to simulate the naive alternative approach to Futhark.  The first code-generator, as far as is possible, does a one-to-one translation of \tt{LFun}s to assembler-like Futhark code, which relies on a Futhark library we have defined.  The second code generator is less straightforward, and kicks in when the first encounters a \tt{Zip} operation which contains lists of functions - something which is inexpressible in Futhark, and which must be transformed before it can be expressed in Futhark.

\subsection{Preprocessing}
In this subsection, we go into details on which exact constructs are reduced to,
or expressed as, other more general constructs, or as compositions of simpler
constructs. We also explain how the structure of composed \tt{Comp}s and \tt{Para}s
can be simplified and standardized, and how some \tt{LFun}s can be combined or
simplified. We also get into the details of how \tt{LFun}s can be expressed as
matrices, how this is done for different and composed \tt{LFun}s.
\subsubsection{Desugaring}
Due to some of the language constructs having complex sematics that can be expressed
as compositions of other simpler constructs, we decided that as a preprocessing step
before code generation, caramelize these constructs before we go on to generate the
Futhark code.
\code{src/Preprocesser.hs}{54}{65}
\begin{itemize}
	\item 55. The caramelization function is structured as a match case, where the input \tt{LFun} is deconstructed in various cases, and the result of the function in each case then corresponds to the result of desugaring.
	\item 56. \tt{KZero} is translated into a left-sectioned outer product with 0, which is the not the most efficient implementation. See the section on future work for more.
	\item 57. \tt{Scale} is likewise translated as \tt{KZero}, although the real number is preserved, as one would expect.
	\item 58---61. \tt{Para}, \tt{Comp}, \tt{LMap} and \tt{Zip} are not changed, but the \tt{LFun}s they compose are recursively desugared.
	\item 62. \tt{Lplus} is expressed as a sequential composition such that a the \tt{Val} to which \tt{Lplus} is applied is first duplicated in a tuple, then the two imbedded \tt{LFun}s are applied to these two values, and the result is then added. This is expressed as a \tt{Dup}, followed by \tt{Para} and then \tt{Add}. The imbedded \tt{LFun}s are also recursively desugared.
	\item 63---65 \tt{Prj} (custom) projections are desugared such that those corresponding to \tt{Fst} and \tt{Snd} are translated into those, and the rest are reported as errors.
\end{itemize}
The rest of the \tt{LFun}s are unchanged, and therefore omitted here.
\subsubsection{Optimization}
We have another preprocessing step, which is optional, and which recursively reshapes and simplifies an \tt{LFun}.
\code{src/Preprocesser.hs}{4}{6}
This is implemented by \tt{optimize}, which calls \tt{optimizeRun} untill it has no effect on the input. This way we make sure that the optimizations are fully effected.

\code{src/Preprocesser.hs}{8}{29}
Two of the optimization cases (l18-19) attempt to regularize sequences of operations (\tt{Comp}) such that all sequences are in a regularized form. This reduction in possible equivalent expressions of the same semantics lead to simpler application of other optimizations.

\begin{itemize}
	\item 11---12. \tt{Id}s in sequence with some other \tt{LFun} can be reduced to that other \tt{LFun}.
	\item 13. Two parallel \tt{Id}s can be more simply expressed as a single \tt{Id}, since it works with all values.
	\item 18. Tree's of sequantial compositions (\tt{Comp}s) which all express the same terminal \tt{LFun}s in the same order can be restructured to also have the same simple shape, which is done in such a way that all left-nodes in the tree are leaf-nodes.
	\item 19. In order to make more optimizations, which are generally applicable to sequences of operations, simpler to apply, equivalent sequences of parallel operations are expressed as parallel sequences of operations.
	\item 20. The above is repeated for the case where the two parallel operations are in a sequence of a different shape.
	\item 23---24. Sequential \tt{Scale} operations are combined, in the two cases relevant in the regularized form, by multiplying the factors at compile-time.
	\item 26---29. The last four cases simply recurse on inner \tt{LFuns}.
\end{itemize}

The rest of the cases (not shown) do not modify the input.


\subsubsection{Linear Maps as Matrices}

The canonical way to represent linear maps on vectors is usually as matrices
with matrix-matrix multiplication as function composition and matrix-vector
multiplication as function application.  Matrices are well-studied, and matrix
operations are used as performance measure for hardware.  For example, GPU manufacturer
NVIDIA\footnote{\url{https://developer.nvidia.com/cublas}} maintains a Basic
Linear Algebra Subroutines (BLAS) library that performs well on their GPUs.

We decided to use matrix computations as a baseline with which we could compare
our symbolic execution, as this is a very common approach in AD. However, simply
expressing linear maps as matrices and multiplying them together is probably very
inefficient. For example, $Id : R^{10^{10}} \rightarrow_1 R^{10^{10}}$ can either be
represented as just a few bytes in a symbolic approach, or it can be represented as a
$R^{10^{20}}$ matrix, the data of which would take up an astronomical 400 exabytes.

So we need a way to go from our existing symbolic
representation to a matrix representation.  This required building an extra
preprocessing step, which is contained in \texttt{src/Matrix.hs}.  A convenient
nature of our project is, that since our DSL already supports a way to express
matrix multiplication, we can write a function with the signature
\texttt{getMatrixRep :: LFun \rightarrow{} LFun} and apply it on any
\texttt{LFUN}.  This results in a new \texttt{LFun} that we can substitute for
the original \texttt{LFun}.  Thus the preprocessing step is completely
transparent for the rest of the compilation chain.

There is a caveat, however.  When dealing with matrices, we need to specify up
front the dimensions of the vector spaces that constitute their domain and
codomain.  In this project we have worked only with endomorphisms, meaning that
the domain and codomain are always the same.  This means that we only need to concern
ourselves with square matrices, and thus it is sufficient to provide a single
integer, specifying the number of {rows} \textit{and} {columns}.

The function signature then becomes \texttt{getMatrixRep :: Int \rightarrow{}
LFun \rightarrow{} LFun}.

\paragraph{Implementation of \texttt{getMatrixRep}}
This method is described in
~\cite[Representing a Linear Map by a Matrix, ch. 3, p. 70]{ladr}.

\code{src/Matrix.hs}{113}{133}

It is important to note that we generate a matrix representation \textit{with
respect to some basis}.  This representation is unique, again
\textit{with respect to that basis}. However, we are only ever concerned with
the \textbf{standard basis}, so this will be implicit in the following.

The method consists of three steps:
\begin{description}
\item [Step 1] Generate the standard basis for the dimensions we want compute
linear map of.  Note again, that we only need one integer here due to the
endomorphic nature of the linear maps used.
\item [Step 2] Apply the linear function in question on each of
the column vectors of the standard basis matrix from the first step.  This
results in a
new set of columns which concatenated gives the desired matrix.
\item [Step 3] Using this matrix together with matrix-vector multiplication,
gives a way to apply the matrix to any vector in the domain.  Essentially
we can wrap it in section notation as: \texttt{LSec Mtx MatrixMultiplication}.
Using matrix-matrix multiplication with other such matrices effectively computes
function composition of those linear maps.
\end{description}

\subsection{Code generation}\label{sec:codegen}
Translating the language into Futhark a lot more complicated than just
evaluating it in the interpreter.  The initial idea is to map \tt{Val}s to
Futhark values, and \tt{LFun}s to Futhark functions.  Whenever an
\tt{LFun} contains a \tt{Val}, this should translate to an equivalent,
partially applied, Futhark function.  We do this by generating one line
of code for each \tt{LFun} - each defining a function, partially applied
or not.

In order to keep the generated code itself as simple as possible, we decided
to define a Futhark library which each generated program imports, and
which contains as much of the logic as possible, such that each of these
generated functions is just the application of one of the functions in the
library.  Because the \tt{LFun}s are simple in themselves, the generated code
comes to resemble three-address code.

In order to compose these functions correctly, whether in parallel or in
sequence, we must give each function a unique name, and the compilation
process must keep track of these names in such a way that the mutual
relations of the \tt{LFun}s is carried over into the generated Futhark
code.

\subsubsection{Arity typing}
Due to the constraints mentioned in the relevant section, our Futhark
generated code and library must not contain any recursion or function
overloading.  However, the \tt{LFun}s themselves are both recursive and
overloaded.  The need to disambiguate these things in the code, created the
need to define and identify the \tt{Arity} of a \tt{Val}, and keep track of
these as well during the compilation.

\code{src/Types.hs}{78}{81}
Recall that the \tt{Tensor} data-type is recursive, as is --- for instance --
the \tt{neg} \tt{LFun}.  In the interpreter, applying \tt{neg} requires one
function with two overloaded definitions - one recursive case (\tt{Tensor})
and one base case (\tt{Scalar}).  Since there is no overloading or recursion
in Futhark, we must have one definition for each level of nestedness for
the \tt{Tensor}s.  Since an \tt{Arity} of \tt{Atom 0} corresponds to a
\tt{Scalar}, \tt{Atom 1} to a vector, \tt{Atom 2} to a matrix, and so on,
identifying these is a requirement to then be able to apply \tt{neg\_0},
\tt{neg\_1} and \tt{neg\_2} and so on as needed.  These are defined in
terms of each other in this manner:

\code{lib/lmaplib.fut}{69}{73}

Arities are identified using the getArity function:

\code{src/Types.hs}{88}{98}

\tt{APair} corresponds to a \tt{Pair}, and contains the arities of the values
of a given pair.  This helps us to ensure that functions can or cant take
pairs are given or not give pairs, among other things.

When generating the code for a single \tt{LFun}, several arities are relevant to
both ensure correctness and to generate meaningful Futhark code: The
arities of all the input values, and the arity of the resulting value.
The arities of the input are given by the embedded value, if it is a
sectioned \tt{LFun}, and by the preceding \tt{LFun} in the Comp chain, or by the input
to the program, if it is the first \tt{LFun} in a chain.  The resulting arity
depends on the \tt{LFun} in question, and is needed to determine that of the
proceeding one in the given chain, or of the resulting program, assuming that
the \tt{LFun} is the {\it{last}} in the given chain.

Since most programs in the intermediate language will involve chains of
sequentially composed \tt{LFun}s, one must, at compile time, know the arity of the
value to which the whole program is applied.  This then disambiguates the
semantics of the first \tt{LFun} in the chain, which then disambiguates the next
one, and so forth.

\subsubsection{The State Monad}
In order to keep track of these things, the Futhark program \tt{String},
the counter used to uniquely identify each \tt{LFun} and the \tt{Arity} of
the \tt{Val} returned from the latest written line of code, during the
compilation process, we decided to use a monad.

The monad just has to contain this state triple of a \tt{Program} (a
\tt{String})

\code{src/Types.hs}{235}{237}

The monad most suited for this purpose is the state monad, which we
implemented this way:

\code{src/CodeGen.hs}{5}{18}

Which is equivalent to a standard implementation for a state monad.  To
interact with this monad, we needed the following auxiliary functions:

\code{src/CodeGen.hs}{20}{27}

\begin{itemize}
	\item 20---21: \tt{put} is used to initialize the state of the monad.
	\item 23---27: \tt{getProg} and \tt{getLastFunIdAndArity} are used to
	extract parts of the state as needed, in a relevant format.
\end{itemize}

\subsubsection{The code generation process}

\code{src/CodeGen.hs}{29}{30}

\begin{itemize}
	\item 29---30: \tt{completeCodeGen} takes the program (\tt{LFun}) the
	user wants to execute, and and example \tt{Val} that the user wants to
	give as input to the program, and compiles it, after having desugared
	and optimized it.  The \tt{Val} is only used to properly arity-type the
	program.  Thus as long as the input has the correct Arity, the program
	will work with.  We do not currently check at compile time whether
	operations such as matrix multiplication will be valid.  This is enforced
	by the Futhark compiler.
\end{itemize}

\code{src/CodeGen.hs}{32}{38}

\begin{itemize}
	\item 32---38: \tt{codeGenProgram} is invoked by \tt{completeCodeGen}.
	It can also be invoked by itself, assuming the \tt{LFun} does not contain
	'sugary' constructs.  The initial state of the monads is the import
	statement for the library, the arity of the input (conceptually, the
	input can be thought of as the output of the preceding program), and the
	constant \tt{0}, which makes function id's start at \tt{1}.  Then
	\tt{codeGenLFun} is called, followed by \tt{finishProg}, and finally the
	source code for the resulting Futhark program is extracted.
\end{itemize}


\code{src/CodeGen.hs}{40}{44}

\begin{itemize}
	\item 40---44: \tt{genLineOfCode} takes the output arity of the function
	being written, the function call itself, and updates the state of the
	monad using these, by composing the complete code line, and updating the
	program, arity and counter.
\end{itemize}

\subsubsection{\tt{Val} translation and Futhark standard input handling}

In order to explain how a program is finished, we must first get into how
\tt{Val}s are translated. When a \tt{Val} is a constant in a program, it
is shown using the show instance:

\code{src/Types.hs}{53}{65}

This means that the \tt{Val} \tt{Pair Dummy (Tensor [Scalar 2, Scalar 3])}
becomes the Futhark literal \tt{(f32.nan, [2.0f32, 3.0f32])}.

However, since Futhark can not accept tuples or pairs via standard
input, we have to deconstruct any tuples used for input and transmit them in
some different way. It happens to be the case, that the 'leaves' in any
'tree' of pairs will always have some mutual ordering, and by ordering these
values in the same way, we can transfer them via standard input without using
pairs. We can then reconstruct the 'tree' in \tt{main} entry point.

\code{src/Types.hs}{68}{75}

We use \tt{stdinShow} whenever the \tt{Val} is used for standard input
rather than a literal in the program. The main difference is that the values
of pairs are printed separately in order, with white space in between. Another difference is that negative numbers have to be encased in parentheses in Futhark literals, but not in standard input.

In order for Futhark to be able to typecheck the generated program, we have
to declare the type of the main entry point. This is not necessarily required
for all Futhark programs in general, but it is sufficient for the compiler to
be able to determine all types in our case.

\code{src/CodeGen.hs}{110}{116}

\tt{genTypeDeclaration} generates the type declaration for a single
standard input value's \tt{Arity}.

\code{src/CodeGen.hs}{119}{122}

In \tt{inputArgDeclaration} each parameter to the \tt{main} entry point is
given a unique name, in a similar way to functions in the main body of the
program, except that this function is not monadic. It generates both the
string corresponding to the declaration, and the string corresponding to the
reconstruction of the input pair tree. These are the two strings in the
return triple, respectively. Please see the code example of a neural network
layer for an example of a result of this in practice.

\code{src/CodeGen.hs}{124}{128}

\tt{finishProg} is the function that actually generates the main entry point.
It must first identify the type declarations and input argument
reconstruction (if applicable), before it can call the last generated main
body function, and append the result to the program.

\subsubsection{\tt{LFun} code generation and Futhark implementations}
The Futhark library begins with these definitions:
\code{lib/lmaplib.fut}{1}{3}

\paragraph{\tt{Id}, \tt{Dup}, \tt{Fst} and \tt{Snd}}
\code{src/CodeGen.hs}{60}{64}
\code{lib/lmaplib.fut}{7}{9}

\begin{itemize}
	\item 60: \tt{codeGenLFun} is the function that disambiguates and
	generates the specific Futhark code for the different \tt{LFun}s.
	It is the 'meat and potatoes' of the code generation process.
	\item 61: \tt{Id} is equivalent to \tt{id}, which is a native Futhark
	function. As one would expect, the output arity (given to the
	\tt{genLineOfCode} function) is the same as the input arity (the
	\tt{a1} argument to \tt{codeGenLFun}).
	\item 62+7: \tt{Dup}'s implementation is self-explanatory. The
	output arity is an \tt{APair} (read: \tt{Arity}-pair) of the input arity.
	\item 63---64+8---9: \tt{Fst} and \tt{Snd} are also very simple, and no
	not warrant long-winded explanation. This time we deconstruct an
	\tt{APair} rather than construct it. Notice how tt{Arity} serves the same
	role as the type system does in Futhark, and can be thought of as
	a 'poor mans type system'.
\end{itemize}


\paragraph{\tt{Add}}

\code{src/CodeGen.hs}{65}{65}

\code{lib/lmaplib.fut}{26}{29}

\begin{itemize}
	\item 65+26---29: The implementation of \tt{Add} assumes that the
	user is adding two values of same \tt{Arity} and length - both of
	these are enforced by Futhark at runtime. The implementation
	itself is very simple, although the type signatures are not. The
	size-type annotations (e.g.. \tt{[n]}) are a necessary promise to the
	Futhark compiler, that the operands are of the same lengths,
	such that the operation will always be successful, as long as the
	types are correct.
\end{itemize}

\paragraph{\tt{Neg}}

\code{src/CodeGen.hs}{66}{66}

\code{lib/lmaplib.fut}{70}{73}

\begin{itemize}
	\item 66---89: \tt{Neg} is a good example of the 'simulated
	recursion' approach we have taken. Each implementation, other
	than the 'base case' just maps over the previous one in the
	'chain'.
\end{itemize}

\paragraph{\tt{Red}}

\code{src/CodeGen.hs}{57}{58}
Due to the constraints of the Futhark size-type system, the compiler
must know at compile time what size the result of the reduce operation
will have. This function determines the highest destination \tt{index}
in the relation, and from this, the size of the resulting output.

\code{src/CodeGen.hs}{68}{68}
The code calling the Futhark function is very simple. We just have
one version of a wrapper-function for each valid arity, and the relation
is passed as a literal value to this function, along with the value we
apply it to.

\code{lib/lmaplib.fut}{86}{93}
Each of the wrappers generate their respective neutral element using
the different \tt{rep} functions, and call the \tt{reduce\_h} function
using the relevant \tt{add} function.

\code{lib/lmaplib.fut}{76}{78}
Each \tt{rep} function replicates a given neutral element. They only vary
in the respective arity of the result.

\code{lib/lmaplib.fut}{80}{84}
The main \tt{reduce\_h} (helper) function is essentially a wrapper for the
native Futhark \tt{reduce\_by\_index} function, which receives a
destination array, an operator with which to perform reduction, a neutral
element for that operator, and two arrays, one of indices and another of
values. It then copies, using the index array, from the value array to the
destination array, using the reduction operator in case there are collisions.
It comparable to a histogram operation, and it is very fast and surprisingly
flexible. It fits with the semantics of our reduce operation, and since it is
native to the Futhark compiler, it is basically impossible to implement
a faster version, with equivalent semantics, using other Futhark
constructs.

\paragraph{\tt{LSec} and \tt{RSec}}
\code{src/CodeGen.hs}{110}{116}
The implementation of sections relies on \tt{bilOpOutputArity} which given the
arities of a bilinear operator (or the loss function) and the arities of the
operands, calculates the arity of the result. It also (currently) enforces
the arities of operands to the various inner product operators to be exactly
as the names suggest. Ideally we would have just one inner product operator,
which was generalized over all these combinations, which would just add the
arities and subtract \tt{2}.

\code{src/CodeGen.hs}{69}{70}
The code for left section is more or less straightforward. The arities and
values are given and written in the same order. For right section, things
get more complicated. Since the function is applied with the arguments
swapped, all the places where the arities are used have to be swapped as
well. Finally, the native Futhark function \tt{flip} must be applied to
the function as well. This does not change the semantics, but only flips the
order in which the arguments are given. \tt{flip} has the (simplified) type \texttt{(a \rightarrow b \rightarrow c) \rightarrow b \rightarrow a \rightarrow c}, which was what we wanted.

\paragraph{\tt{BilOp}s}

\code{src/CodeGen.hs}{46}{55}

\begin{itemize}
	\item 46---52: \tt{bilOp} is used map bilinear operators (and the utility
	operator for calculating loss - this is an implementation detail) to the
	respective Futhark library functions.
	\item 54---55: \tt{arityAnnotation} is used to disambiguate the different
	versions of the given functions.
	\item Thus, calling \tt{bilOp Outer (Atom 2) (Atom 1)} will result in
	\tt{outer\_2\_1} (and its output arity will be \tt{Atom 3}).
\end{itemize}

\code{lib/lmaplib.fut}{39}{40}
The implementation of the generalized outer product relies on these two
functions, \tt{mapr} (map right) and \tt{mapl} (map left). Both of these
map over a single vector, but use operators that take two arguments. Map
right maps over the right argument and passes the left argument to the
operator, and map left maps over the left argument and passes the right
to the operator.

\code{lib/lmaplib.fut}{42}{58}
Each instance of \tt{outer} builds on a previous one, with the 'base case'
being floating point multiplication. Each just 'maps' over either a right
or left vector. It should be easy for the reader to see how these can be
automatically generated.

In the case of the inner products, it is nontrivial to derive a generalized
algorithm which fits all four cases. Nevertheless, all four \tt{BilOp}s
translate to the same underlying Futhark functions. This is a forgiving
implementation, and perhaps a controversial one.

\code{lib/lmaplib.fut}{61}{61}
The dot product of two vectors is the additive reduction of the vector of
their pairwise products.

\code{lib/lmaplib.fut}{62}{62}
Vector matrix multiplication just maps the dot product over the matrix.

\code{lib/lmaplib.fut}{63}{63}
Matrix vector multiplication is the same as vector matrix multiplication,
except that the matrix must first be transposed.

\code{lib/lmaplib.fut}{64}{64}
Matrix matrix multiplication requires mapping over the first argument, and
mapping over the transpose of the second, before their common dimension is
exposed, which then allows application of the dot product function.


\code{lib/lmaplib.fut}{67}{67}
The loss function is the sum of squared errors.

\paragraph{\tt{Comp}}

\code{src/CodeGen.hs}{71}{76}

To compile a sequential composition of two \tt{LFun}s, we must first compile
the first \tt{LFun}, extract its id and output arity, and feed the output
arity as input to the compilation of the second \tt{LFun}. The output arity
of the second becomes the output of the composition, and we then just write
the line of code which chains the two together. This works with arbitrarily
nested \tt{LFun}s, as long as the arity types are correct.

\code{lib/lmaplib.fut}{11}{11}

The Futhark function \tt{comp} then, receives two functions and an
argument, and applies the left function to the result of the right function
on the input, thereby applying them sequentially.

\paragraph{\tt{Para}}

\code{src/CodeGen.hs}{77}{82}

\code{lib/lmaplib.fut}{12}{12}

\tt{Para} is very similar to \tt{Comp}, except that it receives a \tt{Pair}
of \tt{Vals}, and must apply two \tt{LFun}s to the components of the pair in
parallel, and reconstruct a \tt{Pair} containing the result.


\paragraph{\tt{LMap}}

\code{src/CodeGen.hs}{84}{87}

The embedded \tt{LFun} in the \tt{LMap} must first be compiled, and the
value the \tt{LMap} is applied to must be a non-zero \tt{Tensor}. At that
point, \tt{LMap} is equivalent to a Futhark \tt{map}.

\paragraph{\tt{Zip}}

Due to the inadequacy of the main symbolic compiler, in order to compile
\tt{Zip}, we fall back on the secondary compiler, which was developed to
solve this issue. We essentially just apply the alternative compiler to
the subprogram of the \tt{Zip} \tt{LFun}, the result of which is a special
kind of map2.

\code{src/CodeGen.hs}{89}{93}

\code{lib/lmaplib.fut}{14}{14}

\subsubsection{Constant extraction compilation}

As described in the relevant section, \tt{Zip} applies a list of functions
to an equally long list of values. However, in Futhark, functions can
not be stored in arrays. While this would make the problem intractable for
arbitrary lists of functions, we know that \tt{Zip} operations are actually
auto-differentiated maps. This means  that all the functions in the list are
basically the same, they just contain different constant values.


\code{src/Preprocesser.hs}{87}{101}

We introduce the datatype \tt{LFunP} ('pure \tt{LFun}) to denote an
\tt{LFun} which has had its constant \tt{Val}s extracted. We now define
the extraction process in such a way that the constants can be reintroduced
at a later point.

\code{src/Preprocesser.hs}{103}{122}

\begin{itemize}
	\item 105---111: Due to the way the \tt{Val}s are transferred back
	to their original functions, \tt{extractLFunConsts} must give dummy
	values to the \tt{LFun} which do not contain \tt{Val}s.
	\item 112---113: The sectioned functions are the only ones which
	actually contain \tt{Vals}.
	\item 114---120: The values in the embedded \tt{LFun}s in \tt{LMap},
	\tt{Para} and \tt{Comp} are extracted recursively. In the latter two
	cases, \tt{Pair}s are constructed, such that the values in the pairs
	correspond to the embedded \tt{LFun}s.
	\item 121---122: Since we are assuming that the \tt{LFun}s in a
	\tt{Zip} are identical ignoring their constants, when we recursively map
	extract the constants from each inner \tt{LFun}, we only need to keep
	the pure \tt{LFunP} from the first \tt{LFun} in the list. We then
	construct a \tt{Tensor} from the values.
\end{itemize}

\code{src/CodeGen.hs}{140}{149}
The 'const-extracted' compiler, when used standalone, begins very
similarly to the regular symbolic compiler, the main difference being
that this one has to perform the extraction first.

\code{src/CodeGen.hs}{152}{156}
The compilation of the entry point is also very similar. The main difference
being here, that the reconstructed input value pair-tree has to be the
second part of a pair, where the first part consists of the extracted
values pair-tree.

\paragraph{Reconstitution}
\code{src/CodeGen.hs}{159}{165}
\code{src/CodeGen.hs}{167}{167}
\code{lib/lmaplib.fut}{18}{18}

For the \tt{LFunP}s which were given dummy values, we must now wrap the
regular Futhark functions in the \tt{ignoreDummyVal} function, which as the
name suggests, receives a const/arg tuple and throws away the constant
(dummy value) and applies the given function only to the arg.

\code{src/CodeGen.hs}{169}{170}

For the sections, all we have to do is \tt{uncurry} (native in
Futhark) the function such that it can be applied to the const/arg
pair rather than two separate arguments.

\code{src/CodeGen.hs}{172}{177}

\code{lib/lmaplib.fut}{20}{20}

Compiling \tt{CompP} is basically identical to a regular \tt{Comp},
however, the Futhark function is more involved. We are given a const/arg
pair and two functions. The const in the pair is itself a pair of consts,
corresponding to the two functions. We must therefore apply the second (
leftmost) function, with the leftmost constant, to the result of applying
the rightmost function, with the rightmost constant, to the input value.

\code{src/CodeGen.hs}{178}{183}

\code{lib/lmaplib.fut}{21}{21}

Like \tt{CompP}, compiling \tt{ParaP} is basically identical to a regular
\tt{Para}. This time, we are given two functions, a const/arg pair, and
both the const and the arg are themselves pairs. The result is the pair
constructed from applying each function, with their respective constants,
to their respective arguments.

\code{src/CodeGen.hs}{186}{189}

\code{lib/lmaplib.fut}{22}{22}

Due to having to pass the const value to each iteration of the map,
\tt{LMapP} is somewhat different from \tt{LMap}. This is solved by currying
and partially applying \tt{f} to the const value.

\code{src/CodeGen.hs}{191}{195}

\code{lib/lmaplib.fut}{23}{23}

The \tt{ZipP} case is similar to \tt{LMapP}, but now the constant is an
array of constants, so it turns into a \tt{map2}.

