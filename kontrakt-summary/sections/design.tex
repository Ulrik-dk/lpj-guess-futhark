\section{Design of a Linear Map Language}
We will refer to our intermediate language as \tad{}.
In this section, we go through the design of \tad{}. Our function definitions
are based on a draft of~\cite{popl}. A \tad{} program is essentially a linear
function, which can be applied to a real numbered value or some collection of
such. The result of the evaluation of a \tad{} program, whether compiled or
interpreted, is likewise a value. The semantics of \tad{} are kept very limited,
as it is designed to be an intermediate language between two DSLs, and to
maximize the opportunity for automatic code optimizations.

\subsection{Value Types}
The value types of the language are constructed using the following algebraic datatype:

\begin{grammar}
  <Val> ::= <Real>
  \alt Zero
  \alt Vector [<Val>]
  \alt Pair <Val> <Val>
\end{grammar}



A value is therefore either the regular atom Real
(a real number), the special atom Zero, a vector of values or a pair of values.
Because our language is designed for efficient GPU execution, for which regular
parallelism is crucial, vectors must only contain values of the same size, i.e.,
we allow no irregular or mixed vectors.

A pair is a way to pass multiple arguments to parallel functions, and therefore
should not be contained in vectors either. Vectors should only contain atoms or
other vectors, but pairs may contain anything. Pairs can be used to express
arbitrary tensors, by constructing trees. Trees of pairs with identical non-pair
elements in the same absolute ordering (left to right) may be assumed to be
identical, semantically.

A zero represents the zero-vector for whichever vector space it is used it.

\subsection{Linear Functions}\label{linearmaps}
\tad{} has various linear functions which are defined on the values.

\begin{grammar}
<LFun>
  ::= Id
  \alt KZero
  \alt Scale <Real>
  \alt Neg
  \alt LSec <Val> <BilOp>
  \alt RSec <BilOp> <Val>
  \alt LMap <LFun>
  \alt Zip [<LFun>]
  \alt Para <LFun> <LFun>
  \alt Comp <LFun> <LFun>
  \alt Dup
  \alt Prj <Int> <Int>
  \alt Fst
  \alt Snd
  \alt Add
  \alt Lplus <LFun> <LFun>
  \alt Red <Rel>
\end{grammar}

Each \texttt{LFun} can be applied to a value to return a value corresponding to
the result of that application. We also have the following bilinear operators:

\begin{grammar}
<BilOp>
  ::= MatrixMult
  \alt DotProd
  \alt Outer
\end{grammar}

\begin{grammar}
  <Rel> ::= List [(<Int>, <Int>)]
  \alt Func RelFun
\end{grammar}


\subsection{Simple Linear Functions}
The linear function Id is the identity function, and the result of its
application is to leave the input value unchanged. It works for all values.

Scale scales a scalar or vector with a given number. \texttt{KZero} is equivalent to
scaling by zero, and \texttt{Neg} negates all the values in the input. None of these
three are defined for Pairs, and so to for example negate two values in a Pair,
one must compose two \texttt{Neg}s in a \texttt{Para}, and apply this to the Pair of \texttt{Val}s.

\subsection{Compositions} \texttt{Comp $g$ $f$} composes two linear functions
sequentially, such that the right-most \texttt{LFun} is applied first, and then
the left-most, as expected in normal math notation.  Potentially infinitely long
series of \texttt{LFun}s can be composed by using nested \texttt{Comp}
expressions. \texttt{Para} composes two \texttt{LFun}s in parallel, rather than in
sequence. This means that the two given \texttt{LFun}s are applied, each to one
Value in a Pair. For this reason, \texttt{Para} must be applied to a Pair. If one wishes
to apply three functions to three values in parallel, one must compose two Pairs
and two \texttt{Para}s.  Like with pairs, trees of sequential or parallel compositions
can be considered equivalent, as long as the absolute ordering is identical.

The \texttt{LMap} function applies a given \texttt{LFun} to every value of a vector of any
representation, and must therefore be applied to a vector.

The \texttt{Zip} function applies a list of \texttt{LFun}s to each value in a
vector. The lengths must be idential. Because it is
impossible in Futhark to have lists of functions, and because we need to
preserve as much parallelism as possible, our design makes the following
assumption: We know that all \texttt{Zip}s are differentiated Maps at different
points, and therefore with different \texttt{Val} constants. Therefore they all
essentially are of the same function, just with different constants. For this
reason we assume that we can recover the general form and extract the constants.
\texttt{Zip} then becomes a map over an array of parameters and an array of
values. More on this in the CodeGen\autoref{sec:codegen} section. We do not have If functions, but the Fréchet
language does. When an If function is automatically differentiated, it can be
reduced to just either branch. This breaks our assumption, because one branch
will have parameters different to the other. However, if instead of reducing it,
these \enquote{superfluous} non-taken branches are preserved, the general form
will still be shared between all functions, making our implementation compatible
with all auto-differentiated functions.

The \texttt{Dup} function takes a Value and creates a Pair of two copies of the input Value.

The Add function simply adds to Values. This assumes that the two have the same value shape.

\texttt{LPlus} applies one of two \texttt{LFUN}s to each of pair of values and
then adds them. It is considered sugar for sequential composition of \texttt{Para}
followed by Add.

The \texttt{Red} reduce function performs a reduction, and the only relation currently
is that of pairs of source and destination indices.

Scale works for all values except Pairs, and scales

\subsection{Projections} Projections, \texttt{Prj}, \texttt{Fst} and
\texttt{Snd} all extract values from Pairs. \texttt{Fst} returns the first
element, \texttt{Snd} returns the second (rightmost) element. \texttt{Prj}
returns the value at the index of the first integer, from a tuple of size of the
second integer. Currently \texttt{Prj} does not work for tuples of any shape but
2, i.e., a pair, but in the future it could be expanded to be sugar for
compositions of sequential \texttt{Fst} and \texttt{Snd}'s, and tuples could be
implemented as sugar for nested Pairs. I.e., taking the third element would be
taking the second element of the second element of a Pair of a Value and a Pair
of Values, and so on.

\subsection{Bilinear Operators} We currently have three bilinear operators,
which apply bilinear functions to two values---outer product, inner product and
matrix multiplication. The first two are completely generalized for arbitrary
vectors, while the last only works for matrices, or vectors of vectors of
scalars.

The outer product is the pairing of every real value of the first \texttt{Val}
with every real value of the second \texttt{Val}, thereby constructing a new
\texttt{Val} of the products of all these combinations. Scale can be thought of
as the outer product of a simple scalar with some \texttt{Val}---that is
actually what it is compiled to. One way of explaining it is that one takes the
second Value and pairs it with every real number of the first Value, and
performs a scaling of the second Value with each real number of the first, and
inserts the result at the place of the given real number in the first value. If
one is familiar with the concept of a shape array, the resulting shape array is
the appending of the second to the first shape array.

The inner product is an attempt to see the commonalities of the dot product and
matrix multiplication, and to generalize this to any arbitrary combination of
vectors. This means that the inner product, given two matrices, performs matrix
multiplication, and given two vectors, performs dot product.

\subsection{Section Notation} Section notation or simply \enquote{sections} are
partial applications of bilinear operators, where either the left or right value
are given as a constant. The constructors denote which side of the operator the
constant is placed at, meaning that \texttt{LSec} (left section) is given the
left-side argument to the bilinear function, and \texttt{RSec} is given the
right-side argument.

\subsection{Proof of Linearity}
In this subsection, we will prove that our constructs have the property of linearity.

We will proceed with the linear functions the order presented in
\autoref{linearmaps}.  Note that we use ML-like notation where function application
has the highest precedence among all operators.  We will also take the liberty
of using ML-like list notation instead of the more conventional tuple notation
in used in~\cite{ladr}.

\paragraph{Claim:  Programs constructed in TAD are linear by construction}

\textbf{Proof}
We are going to make a proof by cases.  We are focusing on the most challenging
constructs here.

\begin{itemize}

  \item{\textbf{Case} \texttt{Id}:}


\begin{align*}
  \texttt{int Id}\; (ax + by) &= ax+by \\
  &= a(\texttt{Id}\; x) + b(\texttt{Id} \;y).
\end{align*}
Both equalities follows simply from the definition \( \texttt{Id} \; x = x \).

\item{\textbf{Case} \texttt{KZero}:}


\begin{align}
  \texttt{int KZero}\; (ax + by) &= 0 \label{kzero1} \\
  &= a\cdot 0 + b\cdot 0 \\
  &= a(\texttt{KZero}\; x) + b(\texttt{KZero} \;y).\label{kzero3}
\end{align}
Equation~\eqref{kzero1} and~\eqref{kzero3} follow simply from the definition \( \texttt{KZero} \; x = 0 \).

\item{\textbf{Case} \texttt{Scale n}:}


\begin{align}
  \texttt{int (Scale n)}\; (ax + by)
  &= n (a x + b y) \label{scale1} \\
  &= n a x + n b y \label{scale2} \\
  &= a n x + b n y \label{scale3} \\
  &= a(\texttt{(Scale n)}\; x) + b(\texttt{(Scale n)} \;y).\label{scale4}
\end{align}
Equalities~\eqref{scale1} and~\eqref{scale4} follow from the definition \( \texttt{Scale n} \; x = n\cdot x \),
Equality~\eqref{scale2} follows distributivity of scalar multiplication with respect to vector addition.
Equality~\eqref{scale3} follows from commutativity in the underlying field \(F\).

\item{\textbf{Case} \texttt{Neg}:}
This also shows the case for \texttt{Neg} as a special case with \( n = -1 \).

\item{\textbf{Case} \texttt{LMap}:}
Assuming $f$ is a linear map, then \( \texttt{int (LMap $f$)} \colon R^n \rightarrow
R^n\) is also a linear map.


\begin{align*}
  &\phantom{=} \texttt{int (LMap $f$)}\; (a[x_1,\dots,x_n] + b[y_1,\ldots,y_n]) \\
  &= \texttt{int (LMap $f$)}\; [(a x_1 + b y_1), \ldots, (a x_n + b y_n)] \\
  &= [\texttt{$f$}\; (a x_1 + b y_1), \ldots, \texttt{$f$}\; (a x_n + b y_n)] \\
  &= [a \texttt{$f$}\; x_1 + b \texttt{$f$} \; y_1, \ldots, a \texttt{$f$}\;  x_n + b \texttt{$f$}\; y_n] \\
  &= [a \texttt{$f$}\; x_1 ,\ldots, a \texttt{$f$}\; x_n ] +  [b \texttt{$f$}\; y_1 ,\ldots, b \texttt{$f$}\; y_n] \\
  &= a[\texttt{$f$}\; x_1 ,\ldots, \texttt{$f$}\; x_n ] + b [\texttt{$f$}\; y_1 ,\ldots, \texttt{$f$}\; y_n] \\
  &= a\left(\texttt{int (LMap $f$)}\;  [x_1 ,\ldots, x_n ]\right) + b \left( \texttt{int (LMap $f$)}\; [ y_1 ,\ldots, y_n]\right)
\end{align*}


\item{\textbf{Case} \texttt{Zip}:}
Assuming $f_1 \ldots f_n$ are linear maps, then \( \texttt{int (Zip [$f_1,\ldots,f_n$])} \colon R^n
\rightarrow R^n\) is also a linear map.


\begin{align*}
  &\phantom{=} \texttt{int (Zip [$f_1,\ldots,f_n$])}\; (a[x_1,\dots,x_n] + b[y_1,\ldots,y_n]) \\
  &= \texttt{int (Zip [$f_1,\ldots,f_n$])}\; [(a x_1 + b y_1), \ldots, (a x_n + b y_n)] \\
  &= [\texttt{$f_1$}\; (a x_1 + b y_1), \ldots, \texttt{$f_n$}\; (a x_n + b y_n)] \\
  &= [a \texttt{$f_1$}\; x_1 + b \texttt{$f_1$} \; y_1, \ldots, a \texttt{$f_n$}\;  x_n + b \texttt{$f_n$}\; y_n] \\
  &= [a \texttt{$f_1$}\; x_1 ,\ldots, a \texttt{$f_n$}\; x_n ] +  [b \texttt{$f_n$}\; y_1 ,\ldots, b \texttt{$f_1$}\; y_n] \\
  &= a[ \texttt{$f_1$}\; x_1 ,\ldots,  \texttt{$f_n$}\; x_n ] +  b[ \texttt{$f_n$}\; y_1 ,\ldots, \texttt{$f_1$}\; y_n] \\
  &= a \left(\texttt{int (Zip [$f_1,\ldots,f_n$])}\;  [x_1 ,\ldots, x_n ]\right) + b\left( \texttt{int (Zip [$f_1,\ldots,f_n$])}\; [ y_1 ,\ldots, y_n] \right)
\end{align*}


\item{\textbf{Case} \texttt{Comp}:}
Assuming $f$ and $g$ are linear maps, then
\( \texttt{int (Comp $g$ $f$)} \colon V \rightarrow W \) is also linear.


\begin{align}
  \texttt{int (Comp $g$ $f$)}\; (ax + by)
  &= g ( f (a x + b y)) \label{lmcomp1} \\
  &= g ( a f x + b f y) \label{lmcomp2} \\
  &=  a g (f x) + b g (f y) \label{lmcomp3}  \\
  &= a \left( \texttt{int (Comp $g$ $f$)}\; x \right) + b \left( \texttt{int (Comp $g$ $f$)}\; y \right).
\end{align}

Equation~\eqref{lmcomp1} follows from the definition of the function composition
operator, usually written \((g \, \circ \, f)(x)= g(f(x))\).
Equation~\eqref{lmcomp2} holds since \(f\) is linear.
Equation~\eqref{lmcomp3} since \(g\) is also linear.



\item{\textbf{Case} \texttt{Para}:}
Assuming $f$ and $g$ are linear maps, then
\( \texttt{int (Para $f$ $g$)} \colon U \times V \rightarrow W \times X \) is also linear.


\begin{align}
  &\phantom{=} \texttt{int (Para $f$ $g$)}\; ( a(x_1, y_1) + b(x_2, y_2) ) \\
  &= \texttt{int (Para $f$ $g$)}\; (ax_1+bx_2 , ay_1 + b y_2) \\
  &= (f \; (ax_1+bx_2) , g\; ( a y_1 + b y_2)) \\
  &= (af\;x_1+bf\;x_2 , a g\;  y_1 + b g\; y_2) \\
  &= a(f\;x_1 ,  g\;  y_1) + b(f\;x_2 , g\; y_2) \\
  &= a\left( \texttt{int (Para $f$ $g$)}\; (f\;x_1 , g\; y_1) \right)
   + b\left( \texttt{int (Para $f$ $g$)}\; (f\;x_2 , g\; y_2) \right)
\end{align}



\begin{align}
  \texttt{int (Lplus $f$ $g$)}\; (ax + by)
  &= f (a x + b y) + g (a x + b y) \label{lmlp1} \\
  &= ( a f x + b f y) + ( a g x + b g y)  \label{lmlp2} \\
  &= a (f x + g x) + b (f y + g y) \label{lmlp3}  \\
  &= a \texttt{(Lplus $f$ $g$)}\;x + b  \texttt{(Lplus $f$ $g$)}\;y.
\end{align}

Equation~\eqref{lmlp1} follows from the definition of \texttt{LPlus}:  Duplicate the input
so it becomes a pair.  Then apply the functions on the first and second component
respectively.  Finally add the results.  Here we just express the result directly.
Equation~\eqref{lmlp2} holds since \(f\) and \(g\) are linear.
Equation~\eqref{lmlp3} is rearrangement of the terms.

\begin{align}
  &\phantom{=} \texttt{int (Red $r$)}\; (a[t_1,\dots,t_n] + b[z_1,\ldots,z_n]) \\
  &= \texttt{int (Red $r$)}\; [(a t_1 + b z_1), \ldots, (a t_n + b z_n)] \\
  &= \left[
    \sum_{(x,1)\in r} (at+bz)_x
    ,\ldots,
    \sum_{(x,n)\in r} (at+bz)_x
    \right] \\
  &= \left[
    a\sum_{(x,1)\in r} t_x
    +
    b\sum_{(x,1)\in r} z_x
    ,\ldots,
    a\sum_{(x,n)\in r} t_x
    +
    b\sum_{(x,n)\in r} z_x
    \right]
    \\
  &= \left[
    a\sum_{(x,1)\in r} t_x
    ,\ldots,
    a\sum_{(x,n)\in r} t_x
    \right]
  +
    \left[
    b\sum_{(x,1)\in r} z_x
    ,\ldots,
    b\sum_{(x,n)\in r} z_x
    \right]
    \\
  &=a \left[
    \sum_{(x,1)\in r} t_x
    ,\ldots,
    \sum_{(x,n)\in r} t_x
    \right]
  +
    b \left[
    \sum_{(x,1)\in r} z_x
    ,\ldots,
    \sum_{(x,n)\in r} z_x
    \right]
    \\
  &= a\left(\texttt{int (Red $r$)}\;  [t_1 ,\ldots, t_n ]\right) + b \left( \texttt{int (Red $r$)}\; [ z_1 ,\ldots, z_n]\right)
\end{align}

See~\cite[Proposition F.17 \texttt{red} is linear, p. 51]{popl}.


\item{\textbf{Case} \texttt{LSec}:}
Assuming $f_1 \ldots f_n$ are linear maps, then \( \texttt{int (LSec Mtx MatrixMult)} \colon R^n \rightarrow R^n\) is also a linear map.



\begin{align}
  \texttt{int (LSec M MatrixMult)}\; ( aW_1 + bW_2)
  &= M (a W_1 + b W_2)  \\
  &= aMW_1 + bMW_2  \label{lsec1} \\
  &= a\left( \texttt{int (LSec M MatrixMult)}\; W_1 \right) \\
  &+ b\left( \texttt{int (LSec M MatrixMult)}\; W_2 \right)
\end{align}

Equation~\eqref{lsec1} holds because matrix multiplication is linear.

\item{\textbf{Case} \texttt{RSec}:}
  Symmetric to the proof for \texttt{LSec}.

\item Cases for \texttt{Dup}, \texttt{Prj}, \texttt{Fst} and \texttt{Snd} are omitted for
  brevity. %Add can be thought of as LPlus Fst Snd.

\end{itemize}

