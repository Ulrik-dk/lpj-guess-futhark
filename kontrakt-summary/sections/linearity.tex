\subsection{Linear Algebra}

In the following section we are going to introduce basic definitions and
nomenclature for concepts used in linear algebra.  We assume the reader is
familiar with fields and start by introducing vector spaces.  The we introduce
linearity and finally linear map.  These definition will facilitate a more
precise description of our design in the later chapters.

\subsubsection{Vector Spaces}

\paragraph{Definition (Vector Space)}\label{vectorspace}

A set \(V\) over some underlying field \(\pmb{F}\) with associated addition
and scalar multiplication, for which the properties below hold, is called a
vector space.\cite[Def. 1.19, p. 12]{ladr}


\begin{description}

  \item [Commutativity]
    For all \(u,v \in V\) we have \(u+v = v+u \).

  \item [Associativity]
    For all \(u,v,w \in V\) and all \(a,b \in \pmb{F}\) we have both \((u+v)+w = u + (v+w)\)
    and \( (ab)v = a(bv)\).

  \item [Additive identity]
    There exists an element \({0} \in V\) such that for all \(v \in V \) it holds
    that \(v + {0} \).

  \item [Additive inverse]
    Every time we have a \(v \in V\), we must also be able to find a \(w \in V\)
    so that \(v + w = 0 \).

  \item [Mulitiplicative identity]
    The exists an element \({1} \in \pmb{F}\) such that for all \(v \in V\) it
    holds that \({1}v = v\).

  \item [Distributive properties]
    For all \(a,b \in \pmb{F} \) and for all \(u,v \in V\) both
    \(a(u + v) = au + av \) as well as \((a+b)v = av + bv \) must hold.


\end{description}


\subsubsection{Linearity}

In this section we will prove the linearity of each of the constructs that
appear in our DSL.\@   We will then compare the derivation with the actual
implementation of the interpreter to justify that our any program written in
our DSL must also be linear by construction.

\paragraph{Definition (Linearity)} Linearity is a property of a function.  We
say that a function \(f \colon V \rightarrow W\) is linear if for \(x,y \in V\)
and a constant \(a \in \pmb{F}\), it satisfies the following equations:

\begin{description}

\item [Homogeneity] \( f(ax) = af(x) \) The can be read as \enquote{scaling the
  argument and then applying the function is the same as applying the function
    and then scaling the result.}

\item [Additivity] \(f(x+y) = f(x) + f(y) \) This can be understood as \enquote{adding
  two argument and then applying the function is the same as applying the
    function on each of the arguments and then adding the result.}

\end{description}

The combined form \(f(ax+ay) = af(x) + bf(y) \) is true if and only if both of
the above rules hold.  This can sometimes be used to shorten proofs of linearity.

\paragraph{Definition (Linear Map)}\label{linearmap}
A \textit{linear map} is a mapping \(V \rightarrow W\) from a vector space \(V\) to
another \(W\).

\paragraph{Definition (Endomorphism)}\label{endomorphism}
An \textit{endomorphism} of a vector space \(V\) is a linear map from \(V
\rightarrow V\).  That is, a linear map from a vector space \(V\) to itself \(V\).


