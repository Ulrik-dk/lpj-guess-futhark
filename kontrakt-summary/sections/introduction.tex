\section{Introduction}

\paragraph{The Derivative of a Program}   We will start by giving an intuition
for what AD is, and what purpose it serves.  In calculus where we describe the
derivative of a function as the instantaneous change of the function value at
some point.  Analogous to calculus we can consider the derivative of a program.
The easiest way is to picture the program to be a pure computation of its input
variables yielding some result value.  We may then pose the question:  if we
change one of the input variables infinitely little, how does that affect the
result of that program?  A program that takes multiple input variables is
analogous to multivariate functions in calculus.

However, programs can contain structures not found in basic calculus.  For
example:  how do we differentiate control-flow operators such as a
\texttt{for} loop or an \texttt{if-then-else} expression?  The answers are
outside the scope of this project, but we refer to~\cite{adsurvey}.

\cite{adsurvey} describes four different methods for obtaining the
derivative of a program:
\begin{enumerate*}[label=(\arabic*)]

  \item \textbf{Manually computed derivatives} is about first finding the derivative
  by hand and then afterwards implement this it as a separate program.  It does
    not scale to large programs and is prone to human error.

  \item \textbf{Numerical differentiation} uses finite difference
  approximations and while it can be fast it can suffer from precision errors.

\item \textbf{Symbolic differentiation}  applies symbolic manipulation
  rules for finding derivatives, but does so without manual assistance. It can
    suffer from \textit{expression swell}.

  \item \textbf{Automatic differentiation (AD)} is the approach we are concerned with,
  and we will discuss it further in the following chapters.
  AD and symbolic differentiation are equally accurate, but AD does not suffer
    from \textit{expression swell} since it only uses a constant overhead during
    interpretation and it supports control flow.

\end{enumerate*}

From this brief overview we see that AD has properties that makes it
attractive and preferable over other the alternatives.
