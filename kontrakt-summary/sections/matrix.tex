\subsubsection{Linear Maps as Matrices}

The canonical way to represent linear maps on vectors is usually as matrices
with matrix-matrix multiplication as function composition and matrix-vector
multiplication as function application.  Matrices are well-studied, and matrix
operations are used as performance measure for hardware.  For example, GPU manufacturer
NVIDIA\footnote{\url{https://developer.nvidia.com/cublas}} maintains a Basic
Linear Algebra Subroutines (BLAS) library that performs well on their GPUs.

We decided to use matrix computations as a baseline with which we could compare
our symbolic execution, as this is a very common approach in AD. However, simply
expressing linear maps as matrices and multiplying them together is probably very
inefficient. For example, $Id : R^{10^{10}} \rightarrow_1 R^{10^{10}}$ can either be
represented as just a few bytes in a symbolic approach, or it can be represented as a
$R^{10^{20}}$ matrix, the data of which would take up an astronomical 400 exabytes.

So we need a way to go from our existing symbolic
representation to a matrix representation.  This required building an extra
preprocessing step, which is contained in \texttt{src/Matrix.hs}.  A convenient
nature of our project is, that since our DSL already supports a way to express
matrix multiplication, we can write a function with the signature
\texttt{getMatrixRep :: LFun \rightarrow{} LFun} and apply it on any
\texttt{LFUN}.  This results in a new \texttt{LFun} that we can substitute for
the original \texttt{LFun}.  Thus the preprocessing step is completely
transparent for the rest of the compilation chain.

There is a caveat, however.  When dealing with matrices, we need to specify up
front the dimensions of the vector spaces that constitute their domain and
codomain.  In this project we have worked only with endomorphisms, meaning that
the domain and codomain are always the same.  This means that we only need to concern
ourselves with square matrices, and thus it is sufficient to provide a single
integer, specifying the number of {rows} \textit{and} {columns}.

The function signature then becomes \texttt{getMatrixRep :: Int \rightarrow{}
LFun \rightarrow{} LFun}.

\paragraph{Implementation of \texttt{getMatrixRep}}
This method is described in
~\cite[Representing a Linear Map by a Matrix, ch. 3, p. 70]{ladr}.

\code{src/Matrix.hs}{113}{133}

It is important to note that we generate a matrix representation \textit{with
respect to some basis}.  This representation is unique, again
\textit{with respect to that basis}. However, we are only ever concerned with
the \textbf{standard basis}, so this will be implicit in the following.

The method consists of three steps:
\begin{description}
\item [Step 1] Generate the standard basis for the dimensions we want compute
linear map of.  Note again, that we only need one integer here due to the
endomorphic nature of the linear maps used.
\item [Step 2] Apply the linear function in question on each of
the column vectors of the standard basis matrix from the first step.  This
results in a
new set of columns which concatenated gives the desired matrix.
\item [Step 3] Using this matrix together with matrix-vector multiplication,
gives a way to apply the matrix to any vector in the domain.  Essentially
we can wrap it in section notation as: \texttt{LSec Mtx MatrixMultiplication}.
Using matrix-matrix multiplication with other such matrices effectively computes
function composition of those linear maps.
\end{description}
