\paragraph{Abstract}

Auto Differentiation (AD) is the pursuit of computing the derivative of a
program.  The has been the subject of research since the 1960s\cite{adsurvey}.
The \textit{Fréchet language} is a domain specific language (DSL) in which
programs are expressed in such a manner, that a special-purpose interpreter can
derive expressions of the \textit{Fréchet derivative} during interpretation.
However, this approach produces expressions for the derivative that can be
computationally demanding.  Fortunately, this derivative is expressed entirely
in terms of linear maps which are amenable to parallelization.

Futhark\cite{futhark} is a data-parallel language with an optimizing compiler,
that facilitates writing programs that can utilize massively-parallel
hardware.  It produces executables that can then be run on graphics processing
units (GPUs).\@

In this thesis we investigate how we can speed-up computation of these Fréchet
derivatives by utilizing Futhark.\@  We design a DSL suitable for expressing
Fréchet derivatives and implement a translator from this DSL to Futhark
source-code.  We further build an executor that compiles the generated Futhark
source-code using the Futhark compiler and executes it on a GPU.\@  Thereby
enabling us to optimize programs and execute them on a GPU.\@

We provide a test suite to justify the correctness of our implementation, as
well as four micro-benchmarks and an example of a neural network written in the
DSL.\@ Finally, we visualize the performance characteristics over different
input sizes.

\paragraph{Keywords: Auto Differentiation, Fréchet, Futhark, Linear Maps, GPU}


