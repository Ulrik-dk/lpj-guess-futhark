\subsubsection{SOACs}
\paragraph{\tt{map}} is a function which applies a function each value in an
array in parallel. One example could be:

\begin{minted}{haskell}
let pow2 (v: []i32) : []i32 = map (**2) v
\end{minted}

This function raises every element in the given array to the power of two.
Equivalent C-like code might be:

\begin{minted}{C}
void pow2 (int *v, int length) {
	for (int i = 0; i < length; i++){
		v[i] *= v[i];
	}
}
\end{minted}

\paragraph{\tt{reduce}} takes a binary operator, a neutral element and an
array of values, and performs a \it{reduction} on the array. It is more or
less equivalent to a \tt{foldl} in \tt{Haskell}. An example would be:

\begin{minted}{haskell}
let sum (v: []i32) : i32 = reduce 0i32 (+) v
\end{minted}

This function sums up the values in the array and returns the result.
Equivalent C-like code might be:

\begin{minted}{C}
int sum (int *v, int length) {
	int acc = 0;
	for (int i = 0; i < length; i++){
		acc += v[i];
	}
	return acc;
}
\end{minted}


\paragraph{\tt{scan}} takes a binary operator, a neutral element and an
array of values, and performs a \it{prefix sum} on the array. It is similar
to \tt{reduce}. The difference is that all intermediate results are returned.
The resulting array is the same length as the input array.

\begin{minted}{haskell}
let example [n] (v: [n]i32) : [n]i32 = scan (+) 0i32 v
\end{minted}

This function is adds the values in the array and returns the result.
Equivalent C-like code might be:

\begin{minted}{C}
void example (int *v, int ne, int length) {
	int acc = ne;
	for (int i = 0; i < length; i++){
		acc += v[i];
		v[i] = acc;
	}
}
\end{minted}

