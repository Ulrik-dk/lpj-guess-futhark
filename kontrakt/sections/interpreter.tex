\section{Interpreter}
To be able to argue for correctness of our generated Futhark programs, we
decided to implement an interpreter to serve as a reference solution.

With this reference solution, we can argue for the correctness of the much more
complicated and optimized code-generation path.

The interpreter applies an \texttt{LFun} to a \texttt{Val} and returns a wrapped
result.  It follows the usual pattern of a evaluation function, that dispatches
based on the \texttt{LFun} constructor, and makes use of recursive call to
itself with subordinary values.

The wrapped result is a monadic value belonging to the \texttt{Either String a}
monad.  This gives us a number of advantages:

\begin{enumerate}

\item Pure computations and referential transparency.
The function \textit{cannot} have any side effects, and must always
return the same result value on a given input.  Thus

\item Explicit types.  Instead of relying on exceptions during runtime to signal
an error condition, we explicit declare this in the type signature.  We do this
in terms of \texttt{InterpreterResult} which is a simple alias for \texttt{Either
String}.

\item Propagation of descriptive error-messages.  On errors, such
as incompatible values in the constructor, we can propagate back an error
value, that contains a description of exactly what condition was not satisfied.
This is productive for debugging and negative unit tests.

\item We benefit from Haskell's \texttt{do}-notation.  This makes the
interpreter shorter, more readable and easier to reason about compared to other
control-flow structures, such as \texttt{try-catch-finally}.

\end{enumerate}

\paragraph{Values and functions} are represented as the algebraic data types \tt{Val} and \tt{LFun}.
\code{src/Types.hs}{16}{22}
\code{src/Types.hs}{103}{124}

\tt{SparseVector} and \tt{Dummy} are implementation-specific constructors, and not a part of the more theoretical design.Likewise, the injections \tt{InjFst} and \tt{InjSnd} are leftovers from incomplete features.

\paragraph{Id, Dup, Fst and Snd} are implemented as one would expect.

\code{src/Interpreter.hs}{115}{119}
\code{src/Interpreter.hs}{68}{74}

\paragraph{Comp and Para} are likewise.

\code{src/Interpreter.hs}{120}{124}

\paragraph{Sections, bilinear operators, Scale, Neg and KZero}

\code{src/Types.hs}{138}{145}
\code{src/Interpreter.hs}{27}{66}
\code{src/Interpreter.hs}{125}{129}

\paragraph{Add and LPlus}
\code{src/Interpreter.hs}{130}{140}

\paragraph{Reduction} are currently only implemented for source-destination index pairs.

\code{src/Types.hs}{127}{134}
\code{src/Interpreter.hs}{141}{150}
\code{src/Interpreter.hs}{84}{113}

\paragraph{LMap and Zp}
\code{src/Interpreter.hs}{151}{158}
\code{src/Interpreter.hs}{76}{82}
