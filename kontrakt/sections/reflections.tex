\subsection{Reflections} In this subsection we will reflect on selected
practical considerations and difficulties we ran into, that may be relevant for
other students who wish to embark on a similar project.

\paragraph{Platform}
In this project we built a foundation on which we can do experiments.  We made
some experiments on this foundation, but it would certainly be interesting to
do more.  We have prepared to do computations of neural networks with more
layers, and it would be low hanging fruits.  Neural networks are just one
well-studied class of computations, but we would like to test more real-world
examples.

\paragraph{Benchmarking with Futhark}
One of the major feats of engineering was to orchestrate Futhark for data
generation, program compilation, execution and plotting.
We iterated over this three times, because it did not achieve satisfactory
results.  Here we will discuss what we learnt from the earlier iterations.

In the first iteration we generated a self-contained Futhark-program.
Self-contained in the sense, that it contained the randomly generated input
data (Usually a \texttt{Vector : Val}), and the entry point did not take any input.
While it did work in terms of correctness and also in terms of measurements, it took
surprisingly long time to execute.  While the measurements were in the order of
seconds, we waited in the order of 10 minutes in wall clock time.  This turned out
to be caused by the writing the values to the file and then later parsing them
again, the latter of which, we suspect, taking up the larger fraction of the time.
On a positive
note, \texttt{futhark bench} was well-enough engineered that it did not include
parsing time in the measurements it made.

In the second iteration we moved on to using \texttt{futhark dataset} to
generate the data.  We also moved to a binary format, which eliminated the
delays we had experienced.  Since the vectors were random anyway, it did not
affect us much.  Most of the debugging was done on the generated programs and
not on the input data.  Unfortunately the problem returned when we started
generating programs with large matrices involved in section notation, and we did
not find an quick solution for this.  The problem is that the Fréchet
interpreter and the Futhark compiler does not share memory space, so the Fréchet
matrix representation has to be written to disk and then read and parsed by the
Futhark compiler.

\paragraph{Tool chain automation}
We spend some effort on automating our processes es including
correctness testing, benchmarking and parsing and plotting of the results.  Thus
our results plotted in \autoref{plots} are produced by \texttt{stack bench}.

\paragraph{Local GPU}
We benefited a lot from setting tests up early, but one thing that ended up taking
precious time was getting access to a GPU server where we could do benchmarks, as
there were unforeseen issues with our university servers that in the past had worked as
expected. Getting these kinds of things done as early as possible, and setting it up
for easy automation and repetition, will give the student a better idea about how
much time is expendable on the more scientific part of a project. Possibly figuring
out a way to acquire ones own GPU if one does not have such already can help against
externalities.

\paragraph{Data representation} An important early consideration are how data is
represented, as it has huge
implications later on, about how different functions that operate on these will be
implemented. It is well worth thinking this through, as changing it later will be
very costly. A more efficient data representation than the default Haskell list
would have been useful to take care of early in the project.

\paragraph{Generating Futhark code} Generating the code
in as small parts as possible, and then composing them, like assembler, turned out
to be pretty straightforward and work well. Offloading as much as possible to a
library also seems to have been prudent.
