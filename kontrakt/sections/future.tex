\subsection{Future Work}
In the course of the project, we discovered many different tasks that could for one
reason or another not be completed. The most interesting of these are:

\begin{enumerate}
\item Finishing incompletely implemented features like \tt{KZero}, \tt{Zero}, by
adding more relations for \tt{Reduce} and by adding features injections (which
are needed for reverse-mode execution).
\item Investigating more preprocessing optimizations, especially in case of symbolic
tensor products. By reordering the application of functions in ways that do not
change the semantics, but which minimize the computational work-load, it is possible
to achieve significant speedups.
\item Investigating a way to handle conditionals. When conditionals are evaluated in
the automatically differentiating compiler, they are reduced to either branch. Thus
if any \texttt{Zip} contains conditionals, it will not be possible to derive the general form
as we do. An alternative is to not reduce conditionals to the branch taken, but let
the derivative contain both branches and the branch-deciding information, in the
data structure. This will preserve the parallelism of the different derivatives, at
the cost of larger programs, and will allow our approach to handle derivatives of
conditionals.
\item Using QuickCheck for property-based testing of the interpreter and compiler,
by generating random \tt{LFun}s and discovering errors that slip through the cracks
of our unit tests.
\item Expanding the semantics slightly to make them more forgiving, such as treating
trees of \tt{Pair}, and \tt{Para} as semantically flat by always conforming them to
a standard shape, as is already done for \tt{Comp}.
\end{enumerate}

The \tt{Val} \tt{Zero} is partially implemented, and it can therefore be a useful tool in
defining \tt{Val}s, but it is quite a hassle to implement, wherefore we have focused on
other things. Its implementation requires contextual information about the shape of
vectors of various types, which are not otherwise needed, and it breaks some of the
assumptions about the disambiguation of an \texttt{LFun}'s arity, for instance, a program
which just applies the outer product of a value to an input---if the value that is
multiplied with is \texttt{Zero}, then the arity of the program is ambiguous, which is not
allowed in Futhark. All Futhark arrays must have determinable arities, i.e., they may
have runtime-dependent lengths, but the nestedness of arrays must be known.


\paragraph{Improving the duration of the derivative evaluation}
In future work it may be worth looking into solutions to this problem of
redundant store-then-load.  We suggest three ideas to improve upon our work:

\begin{description}

\item [\textbf{Efficient encoding}]  Instead of storing the matrices as encoding
  dense matrices in the program text, it could instead be stored as sparse
    matrices.  This would alleviate the size problem for some matrices, but not
    for all.  It would be fairly easy to implement.

\item [\textbf{Binary encoding}]  Like for the input values, the embedded
  matrices could be encoded in a binary.  This would improve parsing times, at
    the cost of rendering the program texts harder to read and investigate for
    debugging.

\item [\textbf{Shared memory}] The underlying problem is that the two
  compilation processes have isolated memory.  To avoid copying of the
    derivative computation between the address space of Fréchet and Futhark, it
    maybe possible to set up shared memory between them, thus eliminating
    redundant copies of potentially large matrices, but also conversions between
    different data-representations of the same data.

\end{description}

All of these are optimizations may be worth investigating in their own right,
but to truly evaluate their strengths and weaknesses, it would be a good idea to
have a large sample of derivatives from actual Fréchet programs.  This way
the optimizations could be objectively evaluated and directed.

\paragraph{Size Checker}
The arity-based system for checking pre- and postconditions of values used in
and generated from different functions has been sufficient for our implementation.
However, in order to implement \tt{Zero} in Futhark, it would be necessary to
implement something more akin to Futhark's size-type system.
